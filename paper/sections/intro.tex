\section{Introduction}\label{sec:introduction}

Neuroimaging presents fertile grounds for deep learning applications, with myriad interesting real world benefits
for mental health research and disease diagnosis.
High dimensional, massive datasets, with intertwining patterns of activity and structure in three or four
dimensions, are a lot to manage by hand.

Classification of activity and structures should come quite naturally with deep learning practices.
Previous works include the development of Restricted Boltzmann Machines, and Deep Belief Networks~\cite{plis2014deep},
yet often only as a proof of concept, and limited to two dimensions.
Others are working now to try and find universal architectures for most
neuro-imaging tasks~\cite{henschel2019fastsurfer}, though it seems this is still far from a solved
application with satisfactory performances across the board.

Of more interest to myself, deep learning networks in their proven capacity to segment images based on what entities
they identify in classification could surely be adopted to highlight what patterns of activity have informed their
own decision making and perhaps clue researchers in to overlooked details.

Here we'll discuss creating highly accurate classifiers using both 2d and 3d convolutional networks
to learn in the typical manner low level filters, here specifically created for brain scans of spatial activity, using
no pretraining on unrelated datasets and no neuro-imaging feature analysis or reduction~\cite{shi2018feature},
letting the model learn for itself what is important so that it can teach the observer instead.

Following this we'll discuss what this may mean for future works in the highly valuable ability to adapt the classifiers
to the new, related task of unsupervised segmentation in 3 dimensions~\cite{shu2016unsupervised} to mirror what has been
a fruitful field of 2d identification of tumors~\cite{akkus2017deep} for this specific application of telling brain
activity patterns.
